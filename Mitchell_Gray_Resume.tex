\documentclass[letterpaper,11pt]{article}

\usepackage{latexsym}
\usepackage[empty]{fullpage}
\usepackage{titlesec}
\usepackage{marvosym}
\usepackage[usenames,dvipsnames]{color}
\usepackage{verbatim}
\usepackage{enumitem}
\usepackage[colorlinks = true,
            linkcolor = red,
            urlcolor  = Mahogany,
            citecolor = blue,
            anchorcolor = blue]{hyperref}
\usepackage{fancyhdr}
\usepackage[english]{babel}
\usepackage{tabularx}
\usepackage{fontawesome5}
\usepackage{multicol}
\setlength{\multicolsep}{-3.0pt}
\setlength{\columnsep}{-1pt}
\input{glyphtounicode}


%----------FONT OPTIONS----------
% sans-serif
% \usepackage[sfdefault]{FiraSans}
% \usepackage[sfdefault]{roboto}
% \usepackage[sfdefault]{noto-sans}
% \usepackage[default]{sourcesanspro}

% serif
% \usepackage{CormorantGaramond}
% \usepackage{charter}

\hypersetup{pdfborder = 0 0 0}
\pagestyle{fancy}
\fancyhf{} % clear all header and footer fields
\fancyfoot{}
\renewcommand{\headrulewidth}{0pt}
\renewcommand{\footrulewidth}{0pt}

% Adjust margins
\addtolength{\oddsidemargin}{-0.6in}
\addtolength{\evensidemargin}{-0.5in}
\addtolength{\textwidth}{1.19in}
\addtolength{\topmargin}{-.7in}
\addtolength{\textheight}{1.4in}

\urlstyle{same}

\raggedbottom
\raggedright
\setlength{\tabcolsep}{0in}

% Sections formatting
\titleformat{\section}{
  \vspace{-4pt}\scshape\raggedright\large\bfseries
}{}{0em}{}[\color{black}\titlerule \vspace{-5pt}]

% Ensure that generate pdf is machine readable/ATS parsable
\pdfgentounicode=1

%-------------------------
% Custom commands
\newcommand{\resumeItem}[1]{
  \item\small{
    {#1 \vspace{-2pt}}
  }
}

\newcommand{\classesList}[4]{
    \item\small{
        {#1 #2 #3 #4 \vspace{-2pt}}
  }
}

\newcommand{\resumeSubheading}[4]{
  \vspace{-2pt}\item
    \begin{tabular*}{1.0\textwidth}[t]{l@{\extracolsep{\fill}}r}
      \textbf{#1} & \textbf{\small #2} \\
      \textit{\small#3} & \textit{\small #4} \\
    \end{tabular*}\vspace{-7pt}
}

\newcommand{\resumeSpecialSubheading}[6]{
  \vspace{-2pt}\item
    \begin{tabular*}{1.0\textwidth}[t]{l@{\extracolsep{\fill}}r}
      \textbf{#1} & \textbf{\small #2} \\
      \textit{\small#3} & \textit{\small #4} \\
      \textit{\small#5} & \textit{\small #6} \\
    \end{tabular*}\vspace{-7pt}
}

\newcommand{\resumeSubSubheading}[2]{
    \item
    \begin{tabular*}{0.97\textwidth}{l@{\extracolsep{\fill}}r}
      \textit{\small#1} & \textit{\small #2} \\
    \end{tabular*}\vspace{-7pt}
}

\newcommand{\resumeProjectHeading}[2]{
    \item
    \begin{tabular*}{1.001\textwidth}{l@{\extracolsep{\fill}}r}
      \small#1 & \textbf{\small #2}\\
    \end{tabular*}\vspace{-7pt}
}

\newcommand{\resumeSubItem}[1]{\resumeItem{#1}\vspace{-4pt}}

\renewcommand\labelitemi{$\vcenter{\hbox{\tiny$\bullet$}}$}
\renewcommand\labelitemii{$\vcenter{\hbox{\tiny$\bullet$}}$}

\newcommand{\resumeSubHeadingListStart}{\begin{itemize}[leftmargin=0.0in, label={}]}
\newcommand{\resumeSubHeadingListEnd}{\end{itemize}}
\newcommand{\resumeItemListStart}{\begin{itemize}}
\newcommand{\resumeItemListEnd}{\end{itemize}\vspace{-5pt}}



\begin{document}

\begin{center}
    {\Huge \scshape Mitchell Gray} \\ \vspace{1pt}
    \small
    \href{mailto:youremail@gmail.com}{\faEnvelope\ meg346@cornell.edu} ~ 
    \href{https://www.linkedin.com/in/mitchellegray/}{\faLinkedin\ MitchellEGray}  ~
    \href{https://github.com/MitchellGray100}{\faGithub\ MitchellGray100}
    \vspace{-8pt}
\end{center}


% -----------Summary-----------
%-----------EDUCATION-----------
\section{Education}
  \resumeSubHeadingListStart
    \resumeSpecialSubheading
      {Cornell University, Ithaca, NY}{Aug 2020 -- May 2024}
      {Bachelor's of Science, Computer Science}{30 Graduate Credits}%Insert GPA here%
      {Minor, Operations Research Information Engineering}{Deans List: FA 2022, FA 2023, SP 2024}
      %leave a line after this

      \textbf{Relevant Coursework}: Applied High-Performance and Parallel Computing $\cdot$ Distributed Computing $\cdot$ Cloud 
      Computing $\cdot$ Systems Programming $\cdot$ Info Networks $\cdot$ Databases 
      $\cdot$ Operating Systems $\cdot$ Software Testing $\cdot$ AI 
               

  \resumeSubHeadingListEnd
\vspace{-18pt}

%-----------Technical Skilss-----------
\section{Technical Skills}
 \begin{itemize}[leftmargin=0.1in, label={}]
    \small{\item{
     \textbf{Languages}{: Java $\cdot$ C++ $\cdot$ Python $\cdot$ SQL $\cdot$ Bash $\cdot$ C} \\
     \textbf{Software \& Tools}{: 
     CosmosDB $\cdot$ Neo4j $\cdot$ 
     Azure Functions $\cdot$ Azure $\cdot$ Google Cloud $\cdot$ AWS $\cdot$ Kafka $\cdot$
     Git $\cdot$ Protobuf $\cdot$ CI/CD $\cdot$ Docker $\cdot$ Kubernetes
     $\cdot$ Tableau $\cdot$ Qlik $\cdot$ pandas $\cdot$ NumPy $\cdot$
     JavaFX $\cdot$ PyQT $\cdot$ 
     Mockito $\cdot$ JUnit $\cdot$
     Poetry $\cdot$ Maven $\cdot$ CMake}\\
    }}
 \end{itemize}
 \vspace{-20pt}

%-----------EXPERIENCE-----------
\section{Relevant Work Experience}
  \resumeSubHeadingListStart
    \resumeSubheading
      {Oracle}{June 2024 -- Present}
      {Software Developer I $|$ \textbf{\emph{C, SQL, Raft}}}{Redwood City, CA}
      \resumeItemListStart
      \resumeItem {Adding functionality to the Distributed Database at Oracle on the Shard Native Replication team $\cdot$ 
      Utilizing \textbf{Raft} for replication $\cdot$ Writing Distributed Systems code in \textbf{C} $\cdot$ Unit-Testing large-scale software $\cdot$ Using \textbf{SQL} and \textbf{PL/SQL}.}
      \resumeItemListEnd

    \resumeSubheading
      {Gecko Robotics}
      {May 2023 -- Aug 2023}
      {Software Engineer Intern $|$ \textbf{\emph{C++, Python, CMake, C, Google Cloud, CI/CD}}}{Pittsburgh, PA}
      \resumeItemListStart
      \resumeItem {Worked on Robot Controls team $\cdot$ Revamped Robot \& Data Acquisition emulators $\cdot$ 
        Implemented new communications protocol $\cdot$ Wrote code for an asynchronous distributed system $\cdot$ 
        Client/Server TCP networking $\cdot$ Replaced Visual Studio build-system with \textbf{CMake} $\cdot$ Added emulator support
        for calibratable data $\cdot$ Integrated \textbf{Github Actions} and \textbf{Poetry}.}
%Based on the number of point you want to add add the "\resumeItem" for each point you want to add.
      \resumeItemListEnd
    
    \resumeSubheading
      {CMU-Software Engineering Institute }{May 2022 -- May 2023}
      {DevOps Engineer Intern  $|$ \textbf{\emph{Python, Bash, Neo4j, Docker, Kubernetes, CI/CD}}}{Pittsburgh, PA}
      \resumeItemListStart
      \resumeItem {NDA $\cdot$ Updated and created \textbf{Gitlab CI} pipelines $\cdot$ Developed \textbf{Python} 
        and \textbf{Bash} scripts $\cdot$ Created
        \textbf{REST API} data visualizations using \textbf{Qlik} $\cdot$ Used \textbf{ArgoCD} to deploy \textbf{AWS EKS} cluster $\cdot$ Improved               
        efficiency of the company by using \textbf{Python}, \textbf{Neo4j}, \textbf{NeoDash}, and the PageRank algorithm to 
        create useful metrics / long-term documentation $\cdot$ Wrote 30+ page whitepaper using LaTeX $\cdot$ 
        Worked in an agile development environment.}
      \resumeItemListEnd
  \resumeSubHeadingListEnd
\vspace{-16pt}
% adjust vscape based on the space you want to leave between each section.

%------------Research-----------
\section{Research}
  \vspace{-6pt}
      \resumeSubHeadingListStart
      \resumeProjectHeading
      {\textbf{ADOPT: Adaptively Optimizing Attribute Orders} $|$
      \textbf{\emph{Java, AWS}} $|$
      \emph{\href{https://github.com/jxiw/ADOPT}{Github}} $|$
      \emph{\href{https://vldb.org/2023/?papers-demo}{VLDB}}}{Jan 2022 -- Present}
      \resumeItemListStart
              \resumeItem{Paper accepted and presented at \textbf{VLDB 2023} $\cdot$ Currently working
              on transforming the query engine into a \textbf{distributed query engine} $\cdot$ Created dynamic data 
              visualizations for the ADOPT query engine using \textbf{Java}, JavaFX, and GraphStream $\cdot$ Visualized
              the engine's reinforcement learning and worst-case optimal join
              algorithms.}
            \resumeItemListEnd
      \resumeSubHeadingListEnd
  \vspace{-16pt}
% adjust vscape based on the space you want to leave between each section.

%-----------PROJECTS-----------
\section{Projects}
    \vspace{-6pt}
    \resumeSubHeadingListStart
    \resumeProjectHeading
          {\textbf{Distributed Testing Platform (SPEED)} $|$ Masters Project $|$ \textbf{\emph{Java, JUnit, Cloud}\emph{ $|$
          \href{https://github.com/MitchellGray100/SPEED}{Github}}}}{Aug 2023 -- May 2024}
          \resumeItemListStart
            \resumeItem{My team and I created a \underline{S}calable \underline{P}latform for \underline{E}fficient \underline{E}xecution of
            \underline{D}istributed testing. The fault-tolerant system contains a controller node that orchestrates worker nodes that 
            run \textbf{JUnit} tests on \textbf{Java} code. The worker nodes report their findings to the controller. Once all tests are ran, test results are
            shown in the frontend to the user.
            }
          \resumeItemListEnd
          \vspace{-14pt}
    \resumeProjectHeading
          {\textbf{DS-Labs Sharded Key Value Store} $|$ Distributed Computing $|$ \textbf{\emph{Java, Paxos, 2PC}}\emph{ $|$
          \href{https://github.com/emichael/dslabs}{Github}}}{Jan 2023 -- May 2023}
          \resumeItemListStart
            \resumeItem{Using the DS-Lab framework, my partner and I were able to create a sharded transactional key-value 
            store that uses \textbf{Paxos} for replication and \textbf{2PC} for multi-key updates. We implemented
            an Exactly-once RPC protocol on top of an asynchronous network, a primary-backup protocol, 
            Paxos implemented with the PMMC protocol, and 2PC.}
          \resumeItemListEnd
          \vspace{-14pt}
    \resumeProjectHeading
          {\textbf{Cornell Meetup} $|$ Cloud Computing $|$ \textbf{\emph{Azure, CosmosDB, Python, WebDev}}\emph{ $|$
          \href{https://github.com/MitchellGray100/CornellMeetUp}{Github}
          $|$ \href{https://cis.cornell.edu/about/outreach-events/boom-bits-our-minds/projects/boom-2023-projects}
          {BOOM}}}{Aug 2022 -- Dec 2022}
          \resumeItemListStart
            \resumeItem{1 of 32 projects selected for BOOM 2023. My partner and I created a 
            social media web app that allows users to create groups, chat with friends,
            and see where their friends are when on campus. Accounts details were obfuscated and salted.}
          \resumeItemListEnd
          \vspace{-14pt}
    \resumeSubHeadingListEnd
\vspace{-3pt}

%------------Organizations-----------
\section{Organizations}
    \vspace{-6pt}
    \resumeSubHeadingListStart
    \resumeProjectHeading
          {\textbf{Engineering Entertainment Design Club} $|$ \emph{Lead Programmer \& Secretary  $|$
          \href{https://www.linkedin.com/company/cornell-entertainment-engineering-and-design-club/}{Club}}}{Aug 2022 -- Present}
          \resumeItemListStart
            \resumeItem{Working on a cornhole robot in Python and Arduino $\cdot$ Created a robot that makes cocktails $\cdot$ Worked on our website using HTML, CSS, JS, and Bootstrap 
            $\cdot$ Managed projects and mentored all software teams, comprising of 20+ members.}
          \resumeItemListEnd
          \vspace{-14pt}
    \resumeProjectHeading
          {\textbf{Cornell Tradition Fellowship} $|$ \emph{Fellow $|$
          \href{https://scl.cornell.edu/get-involved/cornell-commitment/cornell-tradition}{Fellowship}
          }}{Aug 2020 -- Present}
          \resumeItemListStart
            \resumeItem{Contains $<$ 4\% of all students $\cdot$ Keep good grades
             $\cdot$ Work and volunteer during the school year $\cdot$ Do 100+ hours of each.}
          \resumeItemListEnd
          \vspace{-14pt}
    \resumeSubHeadingListEnd
\vspace{-3pt}

\end{document}
